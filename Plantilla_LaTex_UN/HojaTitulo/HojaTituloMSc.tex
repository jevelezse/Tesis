%\newpage
%\setcounter{page}{1}
\begin{center}
\begin{figure}
\centering%
\epsfig{file=HojaTitulo/escudo,scale=0.8}
\end{figure}
\thispagestyle{empty} \vspace*{0cm} \textbf{\huge
Asociación de variantes en regiones codificantes de genes con datos clínicos en
pacientes colombianos usando minería de datos}\\[5.0cm]
\Large\textbf{Jennifer Vélez Segura}\\[5.0cm]
\small Universidad Nacional de Colombia\\
Facultad de Ingeniería, Departamento de Ing. Sistemas e Industrial\\
Bogotá D.C., Colombia\\
2019\\
\end{center}

\newpage{\pagestyle{empty}\cleardoublepage}

\newpage
\begin{center}
\thispagestyle{empty} \vspace*{0cm} \textbf{\huge
Asociación de variantes en regiones codificantes de genes con datos clínicos en
pacientes colombianos usando minería de datos}\\[2.0cm]
\Large\textbf{Jennifer Vélez Segura}\\[2.0cm]
\small Tesis presentada como requisito parcial para optar al
t\'{\i}tulo de:\\
\textbf{Magister en Bioinformática}\\[2.5cm]
Director(a):\\
Ph.D. Elizabeth León Guzmán\\[2.0cm]
L\'{\i}nea de Investigaci\'{o}n:\\
Minería de datos en Bioinformática\\
Grupo de Investigaci\'{o}n:\\
MIDAS\\[2.5cm]
Universidad Nacional de Colombia\\
Facultad Ingeniería, Departamento de Ing. Sistemas e Industrial\\
Bogotá D.C., Colombia\\
2019 \\
\end{center}

\newpage{\pagestyle{empty}\cleardoublepage}

\newpage
\thispagestyle{empty} \textbf{}\normalsize
\\\\\\%
\textbf{(Dedicatoria)}\\[4.0cm]

\begin{flushright}
\begin{minipage}{8cm}
    \noindent
        Esta tesis esta dedicada a mi familia quienes han sido mi principal apoyo y soporte durante toda mi vida y a mi mejor amiga que en paz descanse Camila Marcela Sanchez Rubio.\\[1.0cm]\\
      \end{minipage}
\end{flushright}

\newpage{\pagestyle{empty}\cleardoublepage}

\newpage
\thispagestyle{empty} \textbf{}\normalsize
\\\\\\%
\textbf{\LARGE Agradecimientos}
\addcontentsline{toc}{chapter}{\numberline{}Agradecimientos}\\\\
A mis amigos Sergio Solano y Julián Cruz quienes me apoyaron, durante todo el proceso de desarrollo del trabajo, al laboratorio Genetix S.A.S quienes donaron la información utilizada en el presente trabajo, a mis compañeras del laboratorio,a mi familia por todo el apoyo y la paciencia.Finalmente a la profesora Elizabeth León Gúzman por la aceptar la dirección del trabajo y prestar todos sus conocimientos para la culminación de este trabajo. \\

\newpage{\pagestyle{empty}\cleardoublepage}

\newpage
\textbf{\LARGE Resumen}
\addcontentsline{toc}{chapter}{\numberline{}Resumen}\\\\
Se realizó una implementación y validación de un pipeline para la idententifiación de variantes a partir de 4813 secuenciados. Se diseño e implemento un modelo de datos para la gestión de la información de las variantes obtenidas y se le adiciono la información clínica disponible para 227 pacientes. Se diseño un modelo de minería de datos basados en procesamiento de lenguaje natural de las historias clínicas y las cuales se les realizo un agrupamiento, una vez obtnidos los grupos se aplicaron reglas de asociación por cada uno de los grupos obtenidos y por dos genes que fueron el CFTR y RB1. Los resultados obtenidos fueron variantes prefiltradas por calidad, una base de datos implementada con la información clínica y las variantes y finalmente se obtuvieron 5 grupos de pacientes con sus reglas de asociación y una caracterización de variantes en CFTR y RB1 en toda la base de datos. Se implemento un modelo de minería que permite caracterizar y asociar la frecuencia de variantes en genes a las características clínicas de los pacientes. \\

\textbf{\small Palabras clave: Secuenciación, variantes, región codificante, minería de datos, agrupamiento, reglas de asociación, modelo de datos, características clíncias}.\\


\textbf{\LARGE Abstract}\\\\
An implementation and validation of a pipeline was carried out to identify variants from 4813 sequenced. A data model for information management was designed and implemented. It is a data mining model in natural language processing of clinical histories and those that are grouped, once you get the groups apply the rules of association for each of the groups and the two genes that were  CFTR and RB1. The results were classified in the database, in the database and in the database. A mining model is implemented that allows to characterize and associate the frequency of the variants in the genes to the clinical characteristics of the patients.\\[2.0cm]
\textbf{\small Keywords: Sequencing, variants, coding region, data mining, clustering, association rules, information systems, clinical characteristics.}\\