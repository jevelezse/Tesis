%\newpage
%\setcounter{page}{1}
\begin{center}
\begin{figure}
\centering%
\epsfig{file=HojaTitulo/escudo,scale=0.8}
\end{figure}
\thispagestyle{empty} \vspace*{0cm} \textbf{\huge
Asociación de variantes en regiones codificantes de genes con datos clínicos en
pacientes colombianos usando minería de datos}\\[5.0cm]
\Large\textbf{Jennifer Vélez Segura}\\[5.0cm]
\small Universidad Nacional de Colombia\\
Facultad de Ingeniería, Departamento de Ing. Sistemas e Industrial\\
Bogotá D.C., Colombia\\
2019\\
\end{center}

\newpage{\pagestyle{empty}\cleardoublepage}

\newpage
\begin{center}
\thispagestyle{empty} \vspace*{0cm} \textbf{\huge
Asociación de variantes en regiones codificantes de genes con datos clínicos en
pacientes colombianos usando minería de datos}\\[2.0cm]
\Large\textbf{Jennifer Vélez Segura}\\[2.0cm]
\small Tesis presentada como requisito parcial para optar al
t\'{\i}tulo de:\\
\textbf{Magister en Bioinformática}\\[2.5cm]
Director(a):\\
Ph.D. Elizabeth León Guzmán\\[2.0cm]
L\'{\i}nea de Investigaci\'{o}n:\\
Minería de datos en Bioinformática\\
Grupo de Investigaci\'{o}n:\\
MIDAS\\[2.5cm]
Universidad Nacional de Colombia\\
Facultad Ingeniería, Departamento de Ing. Sistemas e Industrial\\
Bogotá D.C., Colombia\\
2019 \\
\end{center}

\newpage{\pagestyle{empty}\cleardoublepage}

\newpage
\thispagestyle{empty} \textbf{}\normalsize
\\\\\\%
\textbf{(Dedicatoria)}\\[4.0cm]

\begin{flushright}
\begin{minipage}{8cm}
    \noindent
        Esta tesis esta dedicada a mi familia quienes han sido mi principal apoyo y soporte durante toda mi vida y a mi mejor amiga que en paz descanse Camila Marcela Sanchez Rubio.\\[1.0cm]\\
      \end{minipage}
\end{flushright}

\newpage{\pagestyle{empty}\cleardoublepage}

\newpage
\thispagestyle{empty} \textbf{}\normalsize
\\\\\\%
\textbf{\LARGE Agradecimientos}
\addcontentsline{toc}{chapter}{\numberline{}Agradecimientos}\\\\
A mis amigos Sergio Solano y Julián Cruz quienes me apoyaron, durante todo el proceso de desarrollo del trabajo, al laboratorio Genetix S.A.S quienes donaron la información utilizada en el presente trabajo, a mis compañeras del laboratorio,a mi familia por todo el apoyo y la paciencia.Finalmente a la profesora Elizabeth León Gúzman por la aceptar la dirección del trabajo y prestar todos sus conocimientos para la culminación de este trabajo. \\

\newpage{\pagestyle{empty}\cleardoublepage}

\newpage
\textbf{\LARGE Resumen}
\addcontentsline{toc}{chapter}{\numberline{}Resumen}\\\\
Se realizo una implementación y validación de un pipeline para la idententifiación de variantes a partir de 4813 secuenciados. Se diseño e implemento un modelo de datos para la gestión de la información de las variantes obtenidas y se le adiciono la información clínica disponible para 227 pacientes. Se diseño un modelo de minería de datos basados en procesamiento de lenguaje natural de las historias clínicas y las cuales se les realizo un agrupamiento, una vez obtnidos los grupos se aplicaron reglas de asociación por cada uno de los grupos obtenidos y por dos genes que fueron el CFTR y RB1. Los resultados obtenidos fueron variantes prefiltradas por calidad, una base de datos implementada con la información clínica y las variantes y finalmente se obtuvieron 5 grupos de pacientes con sus reglas de asociación y una caracterización de variantes en CFTR y RB1 en toda la base de datos. Se implemento un modelo de minería que permite caracterizar y asociar la frecuencia de variantes en genes a las características clínicas de los pacientes. \\

\textbf{\small Palabras clave: Secuenciación, variantes, región codificante, minería de datos, clustering, reglas de asociación, sistemas de información, características clíncias}.\\


\textbf{\LARGE Abstract}\\\\
Es el mismo resumen pero traducido al ingl\'{e}s. Se debe usar una extensi\'{o}n m\'{a}xima de 12 renglones. Al final del Abstract se deben traducir las anteriores palabras claves tomadas del texto (m\'{\i}nimo 3 y m\'{a}ximo 7 palabras), llamadas keywords. Es posible incluir el resumen en otro idioma diferente al espa\~{n}ol o al ingl\'{e}s, si se considera como importante dentro del tema tratado en la investigaci\'{o}n, por ejemplo: un trabajo dedicado a problemas ling\"{u}\'{\i}sticos del mandar\'{\i}n seguramente estar\'{\i}a mejor con un resumen en mandar\'{\i}n.\\[2.0cm]
\textbf{\small Keywords: palabras clave en ingl\'{e}s(m\'{a}ximo 10 palabras, preferiblemente seleccionadas de las listas internacionales que permitan el indizado cruzado)}\\