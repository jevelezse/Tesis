%\newpage
%\setcounter{page}{1}
\begin{center}
\begin{figure}
\centering%
\epsfig{file=HojaTitulo/escudo,scale=0.8}
\end{figure}
\thispagestyle{empty} \vspace*{0cm} \textbf{\huge
Asociación de variantes en regiones codificantes de genes con datos clínicos en
pacientes colombianos usando minería de datos}\\[5.0cm]
\Large\textbf{Jennifer Vélez Segura}\\[5.0cm]
\small Universidad Nacional de Colombia\\
Facultad de Ingeniería, Departamento de Ing. Sistemas e Industrial\\
Bogotá D.C., Colombia\\
2019\\
\end{center}

\newpage{\pagestyle{empty}\cleardoublepage}

\newpage
\begin{center}
\thispagestyle{empty} \vspace*{0cm} \textbf{\huge
Asociación de variantes en regiones codificantes de genes con datos clínicos en
pacientes colombianos usando minería de datos}\\[2.0cm]
\Large\textbf{Jennifer Vélez Segura}\\[2.0cm]
\small Tesis presentada como requisito parcial para optar al
t\'{\i}tulo de:\\
\textbf{Magister en Bioinformática}\\[2.5cm]
Director(a):\\
Ph.D. Elizabeth León Guzmán\\[2.0cm]
L\'{\i}nea de Investigaci\'{o}n:\\
Minería de datos en Bioinformática\\
Grupo de Investigaci\'{o}n:\\
MIDAS\\[2.5cm]
Universidad Nacional de Colombia\\
Facultad Ingeniería, Departamento de Ing. Sistemas e Industrial\\
Bogotá D.C., Colombia\\
2019 \\
\end{center}

\newpage{\pagestyle{empty}\cleardoublepage}

\newpage
\thispagestyle{empty} \textbf{}\normalsize
\\\\\\%
\textbf{(Dedicatoria)}\\[4.0cm]

\begin{flushright}
\begin{minipage}{8cm}
    \noindent
        Esta tesis esta dedicada a mi familia quienes han sido mi principal apoyo y soporte durante toda mi vida y a mi mejor amiga que en paz descanse Camila Marcela Sanchez Rubio.\\[1.0cm]\\
      \end{minipage}
\end{flushright}

\newpage{\pagestyle{empty}\cleardoublepage}

\newpage
\thispagestyle{empty} \textbf{}\normalsize
\\\\\\%
\textbf{\LARGE Agradecimientos}
\addcontentsline{toc}{chapter}{\numberline{}Agradecimientos}\\\\
A mis amigos Sergio Solano y Julián Cruz quienes me apoyaron, durante todo el proceso de desarrollo del trabajo, al laboratorio Genetix S.A.S quienes donaron la información utilizada en el presente trabajo, a mis compañeras del laboratorio,a mi familia por todo el apoyo y la paciencia.Finalmente a la profesora Elizabeth León Gúzman por la aceptar la dirección del trabajo y prestar todos sus conocimientos para la culminación de este trabajo. \\

\newpage{\pagestyle{empty}\cleardoublepage}

\newpage
\textbf{\LARGE Resumen}
\addcontentsline{toc}{chapter}{\numberline{}Resumen}\\\\
En esta tesis de maestría se propone un modelo para análisis de variantes en regiones codificantes de genes con datos clínicos en pacientes colombianos usando técnicas de minería de datos. Para ello inicialmente se implementó y validó un  pipeline para la identificación de variantes a partir de 4813 genes de 227 pacientes colombianos, las variantes fueron prefiltradas por calidad y almacenadas en una base de datos relacional. Esta base de datos fue diseñada e implementada con el fin de incluir los datos clínicos. \\

Se propuso un modelo para el análisis textual de historias clínicas usando agrupación, y reglas de asociación para describir cada uno de los grupos encontrados. Adicionalmente, se realizó un análisis puntual de asociación de variantes para los genes CFTR y RB1 dado el indice de variabilidad de cada uno de los genes y de su asociación con enfermedades hereditarias. \\ 

Se obtuvieron como resultados una base de datos implementada, cinco grupos de pacientes con sus variantes asociadas y la asociación de variantes a los genes CFTR y RB1 donde se identificaron polimorfismos y variantes patogénicas, generando un modelo de análisis aplicable a otras poblaciones y escalable a exomas y genomas completos. Se tiene como trabajo futuro implementar la base de datos a un modelo Nosql con datos de exomas y genomas completos ya que la integración de la información es vital para poder diseñar e implementar un modelo de minería de datos.  \\

\textbf{\small Palabras clave: Secuenciación, variantes, región codificante, minería de datos, agrupamiento, reglas de asociación, modelo de datos, características clíncias}.\\


\textbf{\LARGE Abstract}\\\\
An implementation and validation of a pipeline was carried out to identify variants from 4813 sequenced. A data model for information management was designed and implemented. It is a data mining model in natural language processing of clinical histories and those that are grouped, once you get the groups apply the rules of association for each of the groups and the two genes that were  CFTR and RB1. The results were classified in the database, in the database and in the database. A mining model is implemented that allows to characterize and associate the frequency of the variants in the genes to the clinical characteristics of the patients.\\[2.0cm]
\textbf{\small Keywords: Sequencing, variants, coding region, data mining, clustering, association rules, information systems, clinical characteristics.}\\