%\newpage
%\setcounter{page}{1}
\begin{center}
\begin{figure}
\centering%
\epsfig{file=HojaTitulo/escudo,scale=0.8}
\end{figure}
\thispagestyle{empty} \vspace*{0cm} \textbf{\huge
Asociación de variantes en regiones codificantes de genes con datos clínicos en
pacientes colombianos usando minería de datos}\\[5.0cm]
\Large\textbf{Jennifer Vélez Segura}\\[5.0cm]
\small Universidad Nacional de Colombia\\
Facultad de Ingeniería, Departamento de Ing. Sistemas e Industrial\\
Bogotá D.C., Colombia\\
2019\\
\end{center}

\newpage{\pagestyle{empty}\cleardoublepage}

\newpage
\begin{center}
\thispagestyle{empty} \vspace*{0cm} \textbf{\huge
Asociación de variantes en regiones codificantes de genes con datos clínicos en
pacientes colombianos usando minería de datos}\\[2.0cm]
\Large\textbf{Jennifer Vélez Segura}\\[2.0cm]
\small Tesis presentada como requisito parcial para optar al
t\'{\i}tulo de:\\
\textbf{Magister en Bioinformática}\\[2.5cm]
Director(a):\\
Ph.D. Elizabeth León Guzmán\\[2.0cm]
L\'{\i}nea de Investigaci\'{o}n:\\
Minería de datos en Bioinformática\\
Grupo de Investigaci\'{o}n:\\
MIDAS\\[2.5cm]
Universidad Nacional de Colombia\\
Facultad Ingeniería, Departamento de Ing. Sistemas e Industrial\\
Bogotá D.C., Colombia\\
2019 \\
\end{center}

\newpage{\pagestyle{empty}\cleardoublepage}

\newpage
\thispagestyle{empty} \textbf{}\normalsize
\\\\\\%
\textbf{(Dedicatoria)}\\[4.0cm]

\begin{flushright}
\begin{minipage}{8cm}
    \noindent
        Esta tesis esta dedicada a mi familia quienes han sido mi principal apoyo y soporte durante toda mi vida y a mi mejor amiga que en paz descanse Camila Marcela Sánchez Rubio.\\[1.0cm]\\
      \end{minipage}
\end{flushright}

\newpage{\pagestyle{empty}\cleardoublepage}

\newpage
\thispagestyle{empty} \textbf{}\normalsize
\\\\\\%
\textbf{\LARGE Agradecimientos}
\addcontentsline{toc}{chapter}{\numberline{}Agradecimientos}\\\\
A mis amigos Sergio Solano y Julián Cruz quienes me apoyaron, durante todo el proceso de desarrollo del trabajo, al laboratorio Genetix S.A.S quienes donaron la información utilizada en el presente trabajo, a mis compañeras del laboratorio,a mi familia por todo el apoyo y la paciencia. Finalmente a la profesora Elizabeth León Guzmán por la aceptar la dirección del trabajo y prestar todos sus conocimientos para la culminación de este trabajo. \\

\newpage{\pagestyle{empty}\cleardoublepage}

\newpage
\textbf{\LARGE Resumen}
\addcontentsline{toc}{chapter}{\numberline{}Resumen}\\\\
En esta tesis de maestría se propone un modelo para el análisis de variantes en regiones codificantes de genes en pacientes colombianos. Los datos corresponden a 227 pacientes a los cuales se les secuenciaron 4813 genes y sus historias clínicas. Las variantes,filtradas por calidad en cada uno de los pacientes, y las historias clínicas fueron almacenadas en una base de datos relacional.\\

Se diseño e implementó un modelo de analisis que integra tres componentes: Un pipeline para la identificación de variantes; un análisis textual de historias clínicas, usando PLN y agrupación; un modelo de minería de datos usando reglas de asociación sobre las variantes y los grupos de pacientes. El analisis textual tiene como propósito identificar grupos de pacientes con patologías similares, según el contenido de sus historias clínicas como resultado se obtuvieron 5 grupos de pacientes. Las reglas de asociación fueron aplicadas a cada uno de los grupos con el fin de identificar las relaciones de las variantes entre sí y con los grupos de pacientes.  \\

Sobre todo el conjunto de datos se analizaron los genes CFTR y RB1 que tienen un alto indice de variabilidad y preveiamente se han asociado a fibrosis quística y retinoblastoma. A través del modelo se identificaron polimorfismos para el gen CFTR y variantes patogénicas para el RB1, mostrando que los grupos de pacientes pueden asociarse  a las variantes encontradas complementando la interpretación de las variantes presentes en los datos.   \\

\textbf{\small Palabras clave: Secuenciación, variantes, región codificante, minería de datos, agrupamiento, reglas de asociación, modelo de datos, características clínicas}.\\


\textbf{\LARGE Abstract}\\\\
In this thesis of master is proposed  model for the analysis of variants in gene coding regions in Colombian patients is proposed. The data corresponds to 227 patients to whom 4813 genes were sequenced. The variants, filtered by quality in each of the patients, and the clinical histories were stored in a relational database.In this thesis of master is proposed  model for the analysis of variants in gene coding regions in Colombian patients is proposed. The data corresponds to 227 patients to whom 4813 genes were sequenced. The variants, filtered by quality in each of the patients, and the clinical histories were stored in a relational database. \\

An analysis model was designed and implemented that integrates three components: A pipeline for the identification of variants; a textual analysis of medical records, using PLN and clustering; a data mining model using association rules about the variants and groups of patients. The purpose of the textual analysis is to identify groups of patients with similar pathologies, according to the content of their clinical histories, as a result, 5 groups of patients were obtained. The association rules were applied to each of the groups in order to identify the relationships of the variants among themselves and with the groups of patients. \\

The CFTR and RB1 genes, which have a high variability and were previously associated with cystic fibrosis and retinoblastoma, were analyzed in the whole data set. Through the model, polymorphisms were identified for the CFTR gene and pathogenic variants for RB1, showing that groups of patients can be associated to the variants found, complementing the interpretation of the variants present in the data. \\


\textbf{\small Keywords: Sequencing, variants, coding region, data mining, clustering, association rules, information systems, clinical characteristics.}\\