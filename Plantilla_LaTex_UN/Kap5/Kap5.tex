\chapter{Conclusiones y trabajo futuro}

\section{Conclusiones}

\begin{itemize}
	\item La identificación de variantes es uno de los procesos más costosos de implementar a nivel comunicacional, ya que se requiere de la disponibilidad de datos si no también de conocimiento y manejo de computación de alto desempeño.
	
	\item La cantidad de herramientas para identificar variantes y la falta de consensos estándar dificultan la decisión de cuáles son las mejores herramientas y criterios para validar la identificación de variantes.
	
	\item Es necesario desarrollar herramientas que permitan hacer las ejecuciones más rápidas y validas para identificar variantes. 
	
	\item La importancia de gestionar la información clínica y genómica dentro de un mismo sistema de información permite que se pueda almacenar por largos periodos de tiempo la información y  que se puedan utilizar para realizar consultas y análisis. 
	
	\item La selección del gestor de la base de datos debe ser amigable, de fácil manejo y que garantice la seguridad de la información ya que los datos clínicos son datos de alta sensibilidad.
	
	\item La utilización de técnicas de minería permiten realizar análisis alternativos de como es la distribución de las variantes en una población, no solo mirando el contexto del gen, si no el estado alélico de las mismas, la distribución por genero, rangos de edad y su relación con el fenotipo.
	
	\item La aplicación al gen CFTR muestra que los pacientes que tienen una sospecha de variantes patógenicas no se encuentran en regiones codificantes y son variantes distintas a las sinónimas y no sinónimas.
	
	\item Para detectar adecuadamente las variantes del gen CFTR se recomienda hacer secuenciación de gen completo ya que las variantes intronicas también generan un efecto patogénico en este gen.
	
	\item El gen RB1 presento variantes patogénicas donde los pacientes fueron portadores y afectados, demostrando que el modelo aplicado es lo suficientemente robusto para mirar el comportamiento de una variante dentro de una población.
	
	\item La visualización de los resultados de clustering y reglas de asociación, permite generar nuevas preguntas con respecto a los datos obtenidos, para proponer nuevas validaciones experimentales.
	
	\item Se presentó una de las primeras base de datos con información de variantes en la población colombiana a nivel de regiones codificantes.
	
	\item Los grupos obtenidos a pesar de compartir palabras de diagnóstico frecuentes en común como seno y cáncer entre sí a nivel de caracterización de variantes son distintos.
	
	\item En el agrupamiento por pacientes se evidencian dos grandes grupos los cuales son los pacientes que tienen algún tipo de cáncer y los pacientes que sufren algún tipo de síndrome. 
	
	\item Los pacientes que tienen un rango de edad de 0 a 10 años son los que más variantes presentan, pero son los pacientes que menos están representado en las reglas de asociación esto se debe a que hay una alta probabilidad de que sus variantes sean de baja frecuencia. 	
	 
\end{itemize}

\section{Trabajo futuro}

Propuestas de futuras investigaciones:

\begin{itemize}
	\item Aumentar el número de pacientes secuenciados, con el fin de aumentar la búsqueda de patrones de variantes en la población colombiana.
	
	\item Integrar pacientes relacionado como padre, madre e hijo para evaluar los patrones de herencia dentro de la población.
	
	\item Aplicar el mismo modelo de minería utilizando variantes de exomas  y genomas completos para aumentar la cantidad de variantes y su tipo, además de relacionar las variantes intrónicas e intergenicas como posibles variantes causales de enfermedades teniendo en cuenta también el fenotipo de los pacientes.
	
	\item Integrar información de origen regional de los pacientes, para observar la distribución de variantes y las enfermedades por regiones en Colombia.
	
	\item Desarrollar una base de datos NoSQL para integrar la información de variantes proveniente de diversos anotadores, con frecuencias poblacionales y con predicotres de patogenicidad de la variantes y la información clínica, además de dejar abierta la posibilidad de agregar más columnas con nueva información de interés haciéndola la base de datos mas permeable a los cambios que hagan las herramientas de anotación. 
	
	\item Desarrollar un sistema de asignación de variantes, que permita evaluar la frecuencia de cada variante dentro de la población sin necesidad de depender las asignaciones dadas por los anotadores.
	
	\item Modificar el cálculo de frecuencias del algorimo Apriori que permita, seleccionar items frecuentes, poco frecuentes e intermedios sin necesidad de que se calcule las asociaciones para todos los ítems. 

\end{itemize}
